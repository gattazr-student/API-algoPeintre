\documentclass[a4paper]{article}


\usepackage{listings}
\usepackage[english]{babel}
\usepackage[utf8]{inputenc}
\usepackage{amsmath}
\usepackage{graphicx}
\usepackage[colorinlistoftodos]{todonotes}

\title{Projet API : algorithme du peintre}

\author{DUNAND Quentin, GATTAZ Rémi, VIAL-GRELIER Aymeric}

\date{28 novembre 2014}

\begin{document}
\maketitle

\section{Introduction}

Voici le compte rendu du projet "\textit{Algorithme du peintre}". 
Vous trouverez ci-joint à ce compte rendu le répertoire "source" dans lequel se trouveront l'intégralité des sources de ce projet.
\begin{enumerate}
\item Introduction
\item Liste des paquetages
\item Utilisation
\item Evaluation des performances
\item Difficultés rencontrées\todo{A modifier si nécessaire}
\end{enumerate}


\section{Liste des paquetages}
\label{sec:examples}

\subsection{paquetage 1}

\todo[inline, color=blue!40]{To Do}

\subsection{paquetage 2}

\todo[inline, color=blue!40]{To Do}

\section{Comment utiliser l'application}
\subsection{Etape 1 : Compilation du fichier reçu}

Si vous lisez ce document, c'est que vous avez correctement reçu l'archive du projet.
\newline
Créez un dossier temporaire nommé tempProjet. Pour celà effectuer la commande
\begin{lstlisting}
	mkdir ~/tempProjet
\end{lstlisting}
Copiez l'archive dans le répertoire créé.
\begin{lstlisting}
	cp Telechargements/nom_archive.tgz ~/tempProjet
\end{lstlisting}
Décompressez l'archive .tar.gz à l'aide de la commande 
\begin{lstlisting} 
	tar zxvf ~/nom_archive.tgz
\end{lstlisting}
Vous trouverez alors un fichier 
\begin{lstlisting} 
	makefile
\end{lstlisting}
Il vous suffit ensuite d'éffectuer la commande 
\begin{lstlisting} 
	make 
\end{lstlisting}
Pour exécuter le programme compilé il faut ensuite 
\begin{lstlisting} 
	./algopeintre
\end{lstlisting}
Ce programme génère un fichier nommé
\begin{lstlisting} 
	nom_fichier.ps
\end{lstlisting}
Qui est ensuite visualisable.

\section{Tests}
Pour ce projet nous considérons que le point critique à tester est la gestion de la mémoire. \\
Par conséquent notre protocole de test consiste à détécter une éventuelle fuite de mémoire en exécutant 20 fois de suite le programme tout en monitorant la mémoire utilisée afin de déceler une possible augmentation anormale de la mémoire utilisée.


\section{Evaluation des performances}
Afin de tester les performances du programme nous avons élaborer le protocole suivant. Nous rajoutons des instructions ada.Real\_time afin de vérifié le temps d'éxécution des différentes fonctions. Nous sommes conscient du fait que ces valeurs sont fortement liées à la capacité de la machine choisie pour effectuer ces tests. Par conséquent nous effectuons 5 relevés de temps. 
\begin{enumerate}
\item Temps de lecture du fichier .off
\item Temps de récupération des maximums en x,y et z
\item Tems d'allocation mémoire
\item Temps de tri
\item Temps de désallocation
\end{enumerate}
Nous obtenons les résultats suivants pour les tests d'éxécution.



\section{Difficultés rencontrées}
 
\todo{A modifier si nécessaire}











\end{document}